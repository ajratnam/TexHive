\documentclass[journal,twoside]{IEEEtran}

% Essential packages
\usepackage{cite}
\usepackage{amsmath,amssymb,amsfonts}
\usepackage{algorithmic}
\usepackage{graphicx}
\usepackage{textcomp}
\usepackage{xcolor}

% Optional packages depending on your needs
\usepackage{subfigure}
\usepackage{algorithm}

\begin{document}

\title{Your Paper Title}

\author{First~Author,~\IEEEmembership{Fellow,~IEEE,}
        Second~Author,~\IEEEmembership{Member,~IEEE,}
        and~Third~Author,~\IEEEmembership{Member,~IEEE}% <-this % stops a space
\thanks{Manuscript received April 19, 2024; revised August 26, 2024.}
\thanks{F. Author is with the Department of Electrical Engineering, University Name, City, Country (e-mail: author@university.edu).}
\thanks{S. Author is with the Department of Computer Science, University Name, City, Country (e-mail: author2@university.edu).}
\thanks{T. Author is with the Research Institute, Organization Name, City, Country (e-mail: author3@org.com).}}

% The paper headers
\markboth{Journal of \LaTeX\ Templates,~Vol.~XX, No.~XX, MONTH~2024}%
{Author \MakeLowercase{\textit{et al.}}: Paper Title}

% make the title area
\maketitle

\begin{abstract}
Your abstract goes here. IEEE journals typically allow longer abstracts compared to conference papers. The abstract should be self-contained and not require reference to the paper. It should be a concise and comprehensive summary of the paper's content, including the purpose, methods, results, and conclusions.
\end{abstract}

\begin{IEEEkeywords}
keyword1, keyword2, keyword3, keyword4, keyword5
\end{IEEEkeywords}

\section{Introduction}
\IEEEPARstart{T}{his} is how you start the first paragraph of your introduction in IEEE journal format. The first letter is a large capital letter.

The introduction should provide sufficient background information to allow the reader to understand and evaluate the results of the present work without referring to previous publications on the topic.

\section{Related Work}
Your literature review and related work go here. This section should be comprehensive for a journal paper.

\section{System Model}
Describe your system model, theoretical framework, or problem formulation here.

\section{Proposed Method}
Detail your proposed method or approach here. For a journal paper, this should be very thorough.

\section{Mathematical Analysis}
Present your mathematical analysis, proofs, or theoretical contributions here.
\begin{equation}
\label{eq1}
y = Ax + B
\end{equation}

\section{Experimental Results}
Present your experimental results here. Include detailed analysis and comparisons.

\subsection{Experimental Setup}
Describe your experimental setup in detail.

\subsection{Performance Metrics}
Define and explain your performance metrics.

\subsection{Results and Analysis}
Present and analyze your results thoroughly.

% Example of a figure
%\begin{figure}[!t]
%\centering
%\includegraphics[width=2.5in]{figure}
%\caption{Figure caption goes here.}
%\label{fig_example}
%\end{figure}

% Example of a table
%\begin{table}[!t]
%\renewcommand{\arraystretch}{1.3}
%\caption{Table Caption}
%\label{table_example}
%\centering
%\begin{tabular}{|c|c|c|}
%\hline
%Column 1 & Column 2 & Column 3\\
%\hline
%1 & 2 & 3\\
%\hline
%\end{tabular}
%\end{table}

\section{Discussion}
Discuss the implications and significance of your results here. For a journal paper, this should be an in-depth discussion.

\section{Conclusion}
Conclude your paper with a summary of the main points and potential future work. Journal conclusions are typically more detailed than conference papers.

\appendix
\section{Proof of Theorem 1}
Include detailed proofs or additional material in appendices.

% Note: IEEE journal papers often use a different bibliography style
\bibliographystyle{IEEEtran}
\bibliography{references}

\end{document} 