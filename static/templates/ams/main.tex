\documentclass{amsart}

\usepackage{amsmath}
\usepackage{amsthm}
\usepackage{amssymb}
\usepackage{graphicx}
\usepackage{hyperref}

% Theorem environments
\newtheorem{theorem}{Theorem}[section]
\newtheorem{lemma}[theorem]{Lemma}
\newtheorem{proposition}[theorem]{Proposition}
\newtheorem{corollary}[theorem]{Corollary}
\theoremstyle{definition}
\newtheorem{definition}[theorem]{Definition}
\newtheorem{example}[theorem]{Example}
\theoremstyle{remark}
\newtheorem{remark}[theorem]{Remark}

\begin{document}

\title{Your Paper Title}
\author{Author Name}
\address{Department Name\\
         Institution Name\\
         City, State, Country}
\email{email@domain.com}

\begin{abstract}
Your abstract goes here. This should be a concise summary of your mathematical work.
\end{abstract}

\maketitle

\section{Introduction}
Your introduction goes here. State the main results and their significance.

\section{Preliminaries}
Define your notation and basic concepts here.

\begin{definition}
State your definitions here.
\end{definition}

\section{Main Results}
Present your main theorems and their proofs.

\begin{theorem}
State your theorem here.
\end{theorem}

\begin{proof}
Your proof goes here.
\end{proof}

\begin{lemma}
Supporting lemmas go here.
\end{lemma}

\section{Applications}
Discuss applications of your results.

\begin{example}
Provide illustrative examples here.
\end{example}

\section{Conclusion}
Summarize your results and suggest future directions.

\bibliographystyle{amsalpha}
\bibliography{references}

\end{document} 